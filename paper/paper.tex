% For tracking purposes - this is V3.1SP - APRIL 2009

\documentclass{acm_proc_article-sp}

\begin{document}

\title{Modular instrumentation of a monadic interpreter using aspects}

\numberofauthors{2}
\author{
% You can go ahead and credit any number of authors here,
% e.g. one 'row of three' or two rows (consisting of one row of three
% and a second row of one, two or three).
%
% The command \alignauthor (no curly braces needed) should
% precede each author name, affiliation/snail-mail address and
% e-mail address. Additionally, tag each line of
% affiliation/address with \affaddr, and tag the
% e-mail address with \email.
%
% 1st. author
\alignauthor
Florent Marchand de Kerchove\\
       \affaddr{Mines Nantes}\\
       \email{fmdkdd@mines-nantes.fr}
% 2nd. author
\alignauthor
Ismael Figueroa\\
       \affaddr{University of Chile}
}

\maketitle
\begin{abstract}
\end{abstract}

% A category with the (minimum) three required fields
%% \category{H.4}{Information Systems Applications}{Miscellaneous}
%A category including the fourth, optional field follows...
%% \category{D.2.8}{Software Engineering}{Metrics}[complexity measures, performance measures]

%% \terms{Theory}

%% \keywords{ACM proceedings, \LaTeX, text tagging} % NOT required for Proceedings

\section{Introduction}
Describe the problem (use an example).

State your contributions:
\begin{enumerate}
\item
\item
\end{enumerate}

\section{The Problem}

\section{Our idea}

\section{The Details}

\section{Related work}

\section{Conclusions}

\section{Acknowledgments}

% The following two commands are all you need in the
% initial runs of your .tex file to
% produce the bibliography for the citations in your paper.
\bibliographystyle{abbrv}
\bibliography{refs}

%% \thebibliography

%% \balancecolumns

\end{document}
